\chapter{Latent linear models}
\section{Factor Analysis}
\begin{equation}
p(z_i) = N(z_i|\mu_0, \Sigma_0)
\end{equation}

\begin{equation}
p(x_i|z_i,\theta) = N(x_i|Wz_i + \mu, \Psi)
\end{equation}

\begin{equation}
p(x_i|\theta) = 
\int N(x_i|Wz_i +\mu, \Psi)N(z_i|\mu_0, \Sigma_0)dz_i)
= N(x_i|W\mu_0 +\mu, \Psi + W\Sigma_0W^T)
\end{equation}

\begin{equation}
cov[x|\theta] = WW^T + \Psi
\end{equation}

\subsection{隐变量推理}
\begin{equation}
p(z_i|x_i, \theta) = N(z_i|m_i, \Sigma)
\end{equation}

\begin{equation}
\Sigma \triangleq (\Sigma_0^{-1} + W^T\Psi^{-1}W)^{-1}
\end{equation}

\begin{equation}
m_i \triangleq \Sigma_i(W^T\Psi^{-1}(x_i - \mu) + \Sigma_0^{-1}\mu_0) 
\end{equation}

\subsection{混合FA}
\begin{equation}
p(x_i|z_i, q_i = k, \theta) = N(x_i|\mu_k + W_kz_i, \Psi)
\end{equation}

\begin{equation}
p(z_i|\theta) = N(z_i|0, I)
\end{equation}

\begin{equation}
p(q_i|\theta) = Cat(q_i|\pi)
\end{equation}

\section{PCA}
\subsection{损失函数}
\begin{equation}
J(W,Z) = \frac{1}{N}
\sum_{n=1}^N||x_n-\hat x_n||^2
\end{equation}

\begin{equation}
J(W,Z) = ||X-WZ^T||^2
\end{equation}

\subsection{EM for PCA}

\begin{equation}
\widetilde{Z} = (W^TW)^{-1}W^T\widetilde{X}
\end{equation}

\begin{equation}
W = \widetilde{X}\widetilde{Z}^T
(\widetilde{Z}\widetilde{Z}^T)^{-1}
\end{equation}

\subsection{模型选择}
\begin{equation}
E(D, L) = \frac{1}{|D|}\sum_{n=1}^N||x_n - \hat{x}_n||^2
\end{equation}



\section{隐维度选择}
\section{ICA}

\section{因素分析}
\begin{enumerate}
\item 隐变量先验
\begin{equation}
p(z) = N(z|\mu_0, \Sigma_0)
\end{equation}
\item 隐线性模型\\
类比线性模型$y = w^T\phi(x)$和$\Phi = Wt$, 在隐线性模型中,每一个特征
由L个隐变量决定,共有M个特征,因此x的参数是一个M*L的矩阵.
M是特征数,L是隐变量数.
\begin{equation}
p(x|z, \theta) = N(Wz + \mu, \Psi)
\end{equation}
\end{enumerate}

\section{问题}
\begin{enumerate}
\item FA与mixture model的区别?\\
Fa 是隐变量是独立连续型,混合模型中隐变量是
one-hot编码,混合模型相当于是无监督的分类问题,即聚类
问题,服从离散分布。
\item FA中的隐变量z之间是独立的吗?\\
不独立,$W^T*W$是z的协方差矩阵,$W*W^T$是x的协方差矩阵
\item FA与PCA的关系?
PCA是FA的特例?
\item PCA不是唯一的,ICA是唯一的?
\end{enumerate}

